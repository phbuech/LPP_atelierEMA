\documentclass[12pt,a4paper]{beamer}
\usetheme{Montpellier}
\usepackage[utf8]{inputenc}
\usepackage[english]{babel}
\usepackage{amsmath}
\usepackage{amsfonts}
\usepackage{amssymb}
\usepackage{makeidx}
\usepackage{graphicx}
\usepackage{lmodern}
\usepackage{kpfonts}
\usepackage{fourier}
\usepackage[left=2cm,right=2cm,top=2cm,bottom=2cm]{geometry}

\begin{document}

\section{Program start}
\begin{frame}
    \frametitle{Program start}
    ema2wav can be started either
    \begin{itemize}
        \item using the \textbf{GUI}
        \begin{itemize}
            \item via binary (.dmg)
            \item via console
        \end{itemize}
        \item using command-line (see documentation)
        \item by importing as a python module (custom script/notebook/\textbf{Google colab notebook}; see documentation)
    \end{itemize}
    
\end{frame}

\section{Installation}
\begin{frame}
    \frametitle{Installation (Anaconda - script version)}
    \begin{itemize}
        \item open Terminal/Console/Anaconda Prompt
        \item create conda environment:\\
        \texttt{conda create --name ema\_env}
        \item activate the environment:\\
        \texttt{conda activate ema\_env}
        \item install the dependencies:\\
        \texttt{pip install -r path/to/ema2wav/folder/\underline{src/requirements.txt}}
    \end{itemize}
\end{frame}

\section{Run ema2wav}
\begin{frame}
    \frametitle{Run ema2wav}
    \begin{itemize}
        \item for Mac: run ema2wav\_ app.app (see GitHub documentation)
        \item script version (GUI, from terminal, ema2wav src directory assumed):\\
        \texttt{python ema2wav\_app.py}
    \end{itemize}
\end{frame}

\section{User input}
\begin{frame}
    \frametitle{User input}
    \centering
    \includegraphics[scale=0.65]{../../../../../pictures/json_input.png}
\end{frame}

\begin{frame}
    \frametitle{User input - parts of the input file}
    \centering
    \includegraphics[scale=1.725]{../../../../../pictures/json_parts.png}
\end{frame}

\section{Conversion process}
\begin{frame}
    \frametitle{Start of the conversion process (GUI)}
    \centering
    \includegraphics[scale=0.25]{../../../../../pictures/GUI_blank.png}
\end{frame}

\begin{frame}
    \frametitle{Start of the conversion process (manual)}
    \begin{itemize}
        \item from terminal (ema2wav src directory assumed)\\
        \texttt{python convert.py config.json}
        \item as Python module:\\
        \texttt{import ema2wav\_core as ec}\\
        \texttt{config\_file = "/path/to/your/config\_file.json"}\\
        \texttt{ec.ema2wav\_conversion(config\_file)}
    \end{itemize}
\end{frame}

\begin{frame}
    \frametitle{Conversion process}
    \centering
    \includegraphics[scale=0.3]{../../../../../pictures/process.png}
\end{frame}

\begin{frame}
    \frametitle{Conversion process - reshaping of the data}
    \centering
    \includegraphics[scale=0.45]{../../../../../pictures/reshape.png}
\end{frame}

\begin{frame}
    \frametitle{Conversion process - Interpolation}
    \centering
    \includegraphics[scale=0.2]{../../../../../pictures/cubic_spline.png}\\
    {\scriptsize (https://en.wikipedia.org/wiki/Spline\_interpolation#/media/File:Cubic\_splines\_three\_points.svg)}
\end{frame}

\begin{frame}
    \frametitle{Conversion process - GUI approach}
    \centering
    \includegraphics[scale=0.3]{../../../../../pictures/GUI_blank.png}
\end{frame}

\begin{frame}
    \frametitle{Conversion process - GUI approach (2)}
    \begin{itemize}
        \item open the folders containing .pos and .wav files
        \item channel allocation: enter name & channel number
        \item parameters of interest: enter channel name & parameters
        \item select filter (if necessary)
        \item open output folder
        \item start the conversion
    \end{itemize}
\end{frame}

\begin{frame}
    \frametitle{Conversion process - GUI approach (3)}
    \begin{itemize}
        \item config file as documentation
        \item can be used for replicating the conversion process (load CONFIG)
    \end{itemize}
\end{frame}

\begin{frame}
    \frametitle{Conversion process - Google Colab approach}
    \begin{itemize}
        \item \href{colab.research.google.com}{\textbf{\textsc{Google Colab}}}
        \item execute Jupyter Notebooks  online
        \item free & easy-to-use
        \item see example notebook
    \end{itemize}
\end{frame}

\begin{frame}
    \frametitle{Conversion process - Google Colab (1)}
    \begin{itemize}
        \item open the example notebook in Google Colab
    \end{itemize}
    \centering
    \includegraphics[scale=0.15]{../../../../../pictures/GC1.png}
\end{frame}

\begin{frame}
    \frametitle{Conversion process - Google Colab (2)}
    \begin{itemize}
        \item open the files folder
    \end{itemize}
    \centering
    \includegraphics[scale=0.15]{../../../../../pictures/GC2.png}
\end{frame}

\begin{frame}
    \frametitle{Conversion process - Google Colab (3)}
    \begin{itemize}
        \item run the first code snippet
    \end{itemize}
    \centering
    \includegraphics[scale=0.15]{../../../../../pictures/GC3.png}
\end{frame}

\begin{frame}
    \frametitle{Conversion process - Google Colab (4)}
    \begin{itemize}
        \item upload your .pos and .wav files
    \end{itemize}
    \centering
    \includegraphics[scale=0.15]{../../../../../pictures/GC4.png}
\end{frame}

\begin{frame}
    \frametitle{Conversion process - Google Colab (5)}
    \begin{itemize}
        \item upload the GC config file
    \end{itemize}
    \centering
    \includegraphics[scale=0.15]{../../../../../pictures/GC5.png}
\end{frame}

\begin{frame}
    \frametitle{Conversion process - Google Colab (6)}
    \begin{itemize}
        \item run the other code snippets and download converted files
    \end{itemize}
    \centering
    \includegraphics[scale=0.15]{../../../../../pictures/GC6.png}
\end{frame}

\section{Annotation \& Measurements in Praat}
\begin{frame}
    \centering
    {\LARGE Annotation \& Measurements in Praat}
\end{frame}

\subsection{Praat tweaks}
\begin{frame}
    \frametitle{Praat tweaks}
    \begin{itemize}
        \item Display default settings in Praat are not suitable for annotating EMA data
        \item Options for the best annotation experience:
        \begin{itemize}
            \item disable the spectrogram
            \item Change sound scaling in the editor window: \\
            (Sound > Sound scaling... > select 'by window and channel')
            \item Mute channels (if necessary):\\
            (Sound > Mute channels... > (ranges) > enter 2:X)
        \end{itemize}
    \end{itemize}
\end{frame}

\subsection{Annotations}
\begin{frame}
    \frametitle{Landmark annotations}
    \centering
    \includegraphics[scale=0.4]{../../../../../pictures/landmarks.png}\\
    \\
    {\scriptsize Gestural landmarks and intragestural intervals for a typical gestural complex (Tilsen, 2014)}
\end{frame}

\begin{frame}
    \centering
    \includegraphics[scale=0.23]{../../../../../pictures/praat_example.png}\\
\end{frame}


\begin{frame}
    \begin{itemize}
        \item Annotations of gestural landmarks using Point Tiers in Praat
        \item retrieve timing information as usual
        \item measure amplitude information by calling 'Get value at time' applied to the Sound object
        \item see \textbf{measurement\_example.praat}
    \end{itemize}
\end{frame}

\begin{frame}
    \frametitle{Remarks}
    \begin{itemize}
        \item never use the entire file for frequency/intensity/spectral measurements! Instead: extract the the audio track and use that for acoustic measurements
        \item make use of a good documentation of your files
    \end{itemize}
\end{frame}
\end{document}