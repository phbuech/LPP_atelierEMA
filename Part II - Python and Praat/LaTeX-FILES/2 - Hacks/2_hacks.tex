\documentclass[12pt,a4paper]{beamer}
\usetheme{Montpellier}
\usepackage[utf8]{inputenc}
\usepackage[english]{babel}
\usepackage{amsmath}
\usepackage{amsfonts}
\usepackage{amssymb}
\usepackage{makeidx}
\usepackage{graphicx}
\usepackage{lmodern}
\usepackage{kpfonts}
\usepackage{fourier}
\usepackage{hyperref}
\usepackage{tipa}
\usepackage[left=2cm,right=2cm,top=2cm,bottom=2cm]{geometry}
\beamertemplatenavigationsymbolsempty

\begin{document}

\section{Annotations}
\begin{frame}
    \frametitle{Annotations}
    \centering
    \begin{Large}
        Why use IPA for annotations?
    \end{Large}
\end{frame}

\subsection{character encoding schemes}
\begin{frame}
    \frametitle{Encoding}
    \begin{itemize}
        \item IPA symbols in Unicode (latin character, IPA extensions, Phonetic extensions)
        \item Whats the problem?
        \item different character encoding schemes:
        \begin{itemize}
            \item ASCII
            \item UTF-8
            \item UTF-16
            \item UTF-32
        \end{itemize}
    \end{itemize}
\end{frame}

\begin{frame}
    \centering
    \includegraphics[scale=0.4]{../../../../../pictures/ascii.png}
\end{frame}

\begin{frame}
    \centering
    \includegraphics[scale=0.2]{../../../../../pictures/utf16.png}
\end{frame}

\begin{frame}
    \centering
    \includegraphics[scale=0.5]{../../../../../pictures/compare.png}
\end{frame}

\begin{frame}
    \begin{center}
        {\LARGE \c{c}}
    \end{center}
    \begin{itemize}
        \item How many glyphs?
    \end{itemize}
\end{frame}

\begin{frame}
    \begin{center}
        {\LARGE \c{c}}
    \end{center}
    \begin{itemize}
        \item How many glyphs?
        \item U+00E7
        \item c (U+0063) + \c{\quad} (U+0328)
    \end{itemize}

\end{frame}

\begin{frame}
    \begin{itemize}
        \item ASCII and UTF-8 are compatible
        \item UTF-16 is not compatible with UTF-8
        \item but:
        \begin{itemize}
            \item UTF-8 with or without BOM
            \item UTF-16 Big Endian (Mac) vs UTF-16 Little Endian (Windows) (UNIX vs NUXI)
        \end{itemize}
        \item Why is this important?
        \begin{itemize}
            \item archiving of your data
            \item data sharing with collaborators
        \end{itemize}
    \end{itemize}
\end{frame}

\subsection{Praat}
\begin{frame}
    \frametitle{Text encoding in Praat (text writing)}
    \centering
    \includegraphics[scale=0.5]{../../../../../pictures/textwriting.png} 
\end{frame}

\begin{frame}
    \frametitle{Text encoding in Praat (text reading)}
    \centering
    \includegraphics[scale=0.5]{../../../../../pictures/textreading.png} 
\end{frame}

\subsection{Recommendations}
\begin{frame}
    \frametitle{Recommendations}
    \begin{itemize}
        \item uniform encoding
        \item X-Sampa
    \end{itemize}
\end{frame}

\begin{frame}
    \centering
    \includegraphics[scale=0.2]{../../../../../pictures/xsampa.png}
\end{frame}

\subsection{Hack for forced alignment}
\begin{frame}
    \centering
    {\LARGE Hack for forced alignment}
\end{frame}

\begin{frame}
    \centering
    \includegraphics[scale=0.3]{../../../../../pictures/audacity1.png}
\end{frame}

\begin{frame}
    \centering
    \includegraphics[scale=0.3]{../../../../../pictures/audacity2.png}
\end{frame}

\begin{frame}
    \centering
    \includegraphics[scale=0.3]{../../../../../pictures/audacity3.png}
\end{frame}

\begin{frame}
    \centering
    \includegraphics[scale=0.3]{../../../../../pictures/audacity4.png}
\end{frame}

\begin{frame}
    \centering
    \includegraphics[scale=0.3]{../../../../../pictures/audacity5.png}
\end{frame}

\begin{frame}
    \centering
    \includegraphics[scale=0.3]{../../../../../pictures/audacity6.png}
\end{frame}

\begin{frame}
    \centering
    \includegraphics[scale=0.3]{../../../../../pictures/audacity7.png}
\end{frame}

\begin{frame}
    \begin{itemize}
        \item use a script to extract the utterances of interest
        \item have one wav and one text file per utterance
        \item use the forced-alignment system of your choice
    \end{itemize}
\end{frame}

\end{document}
